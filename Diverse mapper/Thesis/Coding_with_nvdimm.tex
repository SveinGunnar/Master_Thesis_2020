%!TeX spellcheck = en_US
\documentclass[12pt,a4paper,USenglish]{article}      % Specifies the document class
\usepackage[utf8]{inputenc}
\usepackage[T1]{fontenc,url}
\usepackage{babel,csquotes,newcent,textcomp}
\usepackage[backend=biber,sortcites]{biblatex}
\usepackage{graphicx}
%\usepackage{listings}

\usepackage{listings}
\usepackage{color}

\definecolor{dkgreen}{rgb}{0,0.6,0}
\definecolor{gray}{rgb}{0.5,0.5,0.5}
\definecolor{mauve}{rgb}{0.58,0,0.82}

\lstset{frame=tb,
  language=Java,
  aboveskip=3mm,
  belowskip=3mm,
  showstringspaces=false,
  columns=flexible,
  basicstyle={\small\ttfamily},
  numbers=left,
  numberstyle=\tiny\color{gray},
  keywordstyle=\color{blue},
  commentstyle=\color{dkgreen},
  stringstyle=\color{mauve},
  breaklines=true,
  breakatwhitespace=true,
  tabsize=3
}

\graphicspath{ {./images/} }

\addbibresource{References.bib}

\title{Methodology}  % Declares the document's title.

%\newcommand{\ip}[2]{(#1, #2)}

\begin{document}             % End of preamble and beginning of text.
%\maketitle                   % Produces the title.
\section{Basic programming with NVDIMM}
% - Creating pmempool
% - Different libraries
% - Opening pool in programm
% - initialzing memory
% - Write memory
% - read memory
\subsection{Introductions}
This chapter will be about how to program with NVDIMM. In order to create code that is using NVDIMM the programmer must chose what library to use and there is many libraries to pick from and all are made for different purpose and have many different types of methods. This chapter exist so other can start on the right track and quickly learn how to use NVDIMM.

The libraries and methods described in this chapter are relevant for NVDIMM devices that supports the libraries created by Intel at pmem.io.

\subsection{Different types of libraries}
\subsubsection{Libpmemobj}
Libpmemobj[19] allows objects to be stored in persistent memory without being torn by interruptions. The objects in question are not class objects one finds in C++, but instead they are variable-sized blocks of data that fall under the term object storage. The object has an object ID that is independent when it comes to location. The changes or updates to these objects are atomic because the library have transactions to make this happen. This library can be used for multithreading and are have been optimized for scaling when it comes to multithreading. The main author also mention that the C++ version of this library is the cleanest and least prone to error compared to all the other libraries[10]. He therefore recommends that programmers should start using this library if they are new to persistent memory programming.

\subsubsection{libpmemblk and libpmemlog}
These libraries are made for specific cases. libpmemblk[17] is used for handling large arrays of persistent memory blocks. The blocks must be larger than 512 bytes in order to work. This library is useful if the program is made to manage a block cache. Libpmemlog[18] is used to append log files. If the program logs a lot of data, it might be better to use libpmemlog in order to avoid going through the traditional file system where most of the time would be spent waiting. Both libraries are built upon the libpmem library, but unlike libpmem their updates can’t be torn because of interruptions.

\subsection{Creating pmempool}
\label{section:memorypool}
%pmempool create --layout my_layout --size=50G obj pool.obj
Before NVDIMM can be used, the user must create whats called a memory pool on the NVDIMM. The NVDIMM has several modes, in order to be able to create a memory pool the mode must be set to fsdax. On a server this must be done by the system administrator. To see what mode the NVDIMM is in can be done by the command ndctl-list. A program called pmempool must also be installed on the server, it is this program that will create the memory pool.
The command used for creating the memory pool for this thesis is 
\begin{lstlisting}
pmempool create --layout Layout_name --size=170G obj pool.obj
\end{lstlisting}
The layout is a string stored in the memory pool. When a program access a memory pool it need to send a string that match the string in the memory pool in order to use it. The user can specify the size of the memory pool, if size is not specified the pmempool will create a pool with the lowest size allowed.
There are three different types of memory pool to choose from, they are obj, log and blk. Which type of memory pool to use depends on which type of library is used in the program. In this thesis the libpmemobj library was used and that is why obj was used in the creating of memory pool. For log and blk are for the libraries libpmemlog and libpmemblk.
The last part of the command line is the name and file address of the memory pool.

\subsection{Methods}
In this project it is the libpmemobj library that will be used. Below is a short description of all the methods that will be used in the thesis. 
\subsubsection{Open memory pool}
\label{subsection:openmemorypool}
The memory pool uses a pointer called PMEMobjpool. The memory pool is opened by using the method pmemobj\_open that needs two arguments. The first argument is the path to the memory pool created in chapter \ref{section:memorypool}. The second argument is a text string that identifies what data belongs to what program.
\begin{lstlisting}
PMEMobjpool *pop = pmemobj_open(path, LAYOUT_NAME);
\end{lstlisting}

\subsubsection{Declaring an array}
When the programmer wants to declare an array, the follow command must be used.
\begin{lstlisting}
TOID(Type) Array_name;
\end{lstlisting}
Type is the data type the programmer wants to use and the name is the name of the array.

\subsubsection{Allocating array}
When allocating the array the method called POBJ\_ALLOC is used. The method have six arguements. The first argument is the memory pool created in chapter \ref{subsection:openmemorypool}. The second argument is the the array the user want to allocate memory. Third argument is the data type and the fourth argument is the array length in bytes. The last two arguments is irrelevant in this context and can be given the value NULL.\\
\begin{lstlisting}
//Allocating of the array
POBJ_ALLOC(Memory_pool, &Array_name, Type, sizeof(double) * ARRAY_LENGTH, NULL, NULL);
//Deallocating of the array
POBJ_FREE(&Array_name);
\end{lstlisting}
When deallocating the array the method POBJ\_FREE that only need the array the user want to deallocate as argument.

\subsubsection{Read/Write to array}
%When reading and writing to an array, the following command are used
Reading from the array is done by using the method D\_RO that must have the array the user want to read from as argument. The method also uses square brackets after the argument that need the index of the element in the array the user want to read. The method D\_RW is used when reading to the array. The use of this method is identical to D\_RO.
\begin{lstlisting}
//Reading an array variable.
var = D_RO(Array_name)[index];

//Writing to an array variable.
D_RW(Array_name)[index] = var;
\end{lstlisting}



\subsection{Coding example}
This is an example on how to use the the pmemobj library. The example will find the average of an array where the array is replaced with an NVDIMM array. The purpose is to show how easy it is to code with NVDIMM by having all the relevant methods in an easy example.
The way of using the a NVDIMM library is a lot similar to using ordinary arrays. Once the programmer have chosen what NVDIMM library to use and included the library in the code the memory pool must be opened. The first thing the code need is the path to the memory pool and a layout, which is a string that identifies the pool that the user can choose what it will be. This can either be a command line argument the user gives when starting the program or it can be hard coded into the code. This is what has been done at listing \ref{nvdimm_example} at line 5 and 11. These two strings are used as arguments when initiating the pool at line 14-15. The initiation is also followed up with an if-sentence at line 16-19 to check that the memory pool has been successfully created. If it has not the program will print out an error message and exit the program.

Next is to create a NVDIMM array, this is what happens at line 21. The NVDIMM array pointer is a void pointer that is casted to a double pointer. 
The array gets initiated at line 22. The method used is called POBJ\_ALLOC, this method are similar to malloc for DRAM. The method have six arguments, the first argument is what memory pool the array will be assigned to.
The second argument is the the address of the pointer. Third argument is the type of the elements in the array. The fourth argument is the length of the array, that is the size of type multiplied with the number of elements in array. The last two arguments are set to NULL.

When writing to an NVDIMM array the programmer must use the method called D\_RW. It only have one argument which is the name of the array. It is followed up with a pair square brackets that contains the index of the element the programmer wants to write to, an example can be found at line 25.

D\_RO is the name of the method one must use to read an element from an NVDIMM array. This method also have one argument which is the name of the array and the square brackets contains the index of the element that will be read. Line 29 is an example of how to add the value of an element in a NVDIMM array to a variable.

In order to deallocate the a NVDIMM array one must use the method POBJ\_FREE and the only argument needed is the address of the \\NVDIMM pointer, an example can be found in line 34. If the programmer forgets to free up the NVDIMM array there will be a permanent memory leak that will last even after the program have stopped running. In order to get rid of the memory leak one must delete the memory pool and create a new one.

Lastly one must close the memory pool before the program is terminated. This is done with pmemobj\_close, it only have the pointer for the memory pool as argument. Line 35 shows how to close the memory pool.

%POBJ_ALLOC(PMEMobjpool *pop, TOID *oidp, TYPE, size_t size, pmemobj_constr constructor, void *arg)
\begin{lstlisting}[caption={Example of coding with NVDIMM}, label={nvdimm_example}]
#include <stdio.h>
#include <stdlib.h>
#include <libpmemobj.h>

POBJ_LAYOUT_BEGIN(array);
POBJ_LAYOUT_TOID(array, double);
POBJ_LAYOUT_END(array);

#define ARRAY_LENGTH 1000
#define LAYOUT_NAME "my_layout"
int main(int argc, char *argv[])
{
	double average = 0.0;
	int i;
	//The path for the memory pool.
	const char path[] = "/mnt/pmem0-xfs/pool.obj";
	
	/* create the pmemobj pool or open it if it already exists */
	PMEMobjpool *pop;
	pop = pmemobj_open(path, LAYOUT_NAME);
	if (pop == NULL) {
		perror(path);
		exit(1);
	}
	//Creation of NVDIMM array.
	TOID(double) nvm_array;
	POBJ_ALLOC(pop, &nvm_array, double, sizeof(double) * ARRAY_LENGTH, NULL, NULL);
	//Writing to the array.
	for(i=0;i<ARRAY_LENGTH;i++){
		D_RW(nvm_array)[i] = i;
	}
	//Reading from the NVDIMM array.
	for(i=0;i<ARRAY_LENGTH;i++){
		average += D_RO(nvm_array)[i];
	}
	average = average / ARRAY_LENGTH;
	printf("%f\n", average);
	
	POBJ_FREE(&nvm_array);
	pmemobj_close(pop);
	return 0;
}
\end{lstlisting}
\end{document}
